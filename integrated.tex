\documentclass[11pt]{article}
\usepackage[spanish]{babel}
\usepackage[utf8]{inputenc}

\title{LICENCIA P\'{U}BLICA GENERAL DE GNU}
\date{Versi\'{o}n 3, 29 de junio de 2007}

\begin{document}
\maketitle

\textit{Esta es una traducci\'{o}n no oficial al espa\~nol de la GNU General Public License. No ha sido publicada por la Free Software Foundation, y no establece legalmente las condiciones de distribuci\'{o}n para el software que usa la GNU GPL --estas condiciones se establecen solamente por el texto original, en ingl\'{e}s, de la GNU GPL. Sin embargo, esperamos que esta traducci\'{o}n ayude a los hispanohablantes a entender mejor la GNU GPL. }

\vspace{0.4cm}

\textit{This is an unofficial translation of the GNU General Public License 
into Spanish. It was not published by the Free Software Foundation, and 
does not legally state the distribution terms for software that uses the 
GNU GPL --only the original English text of the GNU GPL does that. However, 
we hope that this translation will help Spanish speakers understand the 
GNU GPL better.}

\begin{center}
{\parindent 0in

Copyright \copyright\  2007 Free Software Foundation, Inc. \texttt{http://fsf.org/}
  
 
\bigskip
Se permite la copia y distribuci\'{o}n de copias literales de este documento, pero no se permite su modificaci\'{o}n. }

\end{center}

\renewcommand{\abstractname}{Pre\'{a}mbulo}
\begin{abstract}
La Licencia P\'{u}blica General de GNU es una licencia libre, bajo ``copyleft'', para software y otro tipo de obras.

Las licencias para la mayor\'{i}a del software y otras obras de 
car\'{a}cter pr\'{a}ctico est\'{a}n dise\~nadas para privarle de la libertad 
de compartir y modificar las obras. Por el contrario, la Licencia P\'{u}blica General de GNU pretende garantizar su libertad de compartir y modificar 
todas las versiones de un programa --para cerciorar que permanece como 
software libre para todos sus usuarios. Nosotros, la Free Software Foundation, usamos la Licencia P\'{u}blica General de GNU para la mayor\'{i}a 
de nuestro software; la cual se aplica tambi\'{e}n a cualquier otra 
obra publicada de esta forma por parte de sus autores. Usted tambi\'{e}n 
puede aplicarla a sus programas.

Cuando hablamos de software libre (free software), nos referimos a libertad, no a precio. Nuestras Licencias P\'{u}blicas Generales est\'{a}n dise\~nadas para garantizar su libertad de distribuir copias de software libre (y cobrar por ellas si lo desea), recibir el c\'{o}digo fuente o poder obtenerlo si quiere, modificar el software o usar fragmentos de \'{e}l en sus nuevos programas, y que sepa que puede hacer esas cosas.

Para proteger sus derechos, necesitamos impedir que otros le denieguen 
esos derechos o que le pidan que renuncie a ellos. Por ello, tiene 
ciertas responsabilidades si distribuye copias del software, o si lo 
modifica: la responsabilidad de respetar la libertad de otros.

Por ejemplo, si distribuye copias de un programa, bien sea gratis o 
por una tasa, debe transferirles a los que lo reciban las mismas 
libertades que usted recibi\'{o}.  Debe asegurarse que ellos, tambi\'{e}n, 
reciben o pueden obtener el c\'{o}digo fuente. Y debe mostrarles estos 
t\'{e}rminos para que ellos puedan conocer sus derechos.

Los desarrolladores que usan la GNU GPL protegen tus derechos con dos 
pasos:
(1) haciendo valer el derecho de propiedad intelectual en el software, y 
(2) ofreci\'{e}ndole esta Licencia que le da el permiso legal para copiarlo, 
distribuirlo y/o modificarlo.

Para la protecci\'{o}n de autores y desarrolladores, la GPL explica 
claramente que no hay garant\'{i}a para este software libre. Por el bien 
tanto de usuarios como de autores, la GPL requiere que las versiones 
modificadas sean marcadas como con cambios, de forma que sus problemas 
no puedan ser atribuidos de forma err\'{o}nea a autores de versiones previas.

Algunos dispositivos est\'{a}n dise\~nados para denegar a los usuarios el 
acceso para instalar o ejecutar versiones modificadas del software en su 
interior, a pesar de que el fabricante puede hacerlo. Esto es 
fundamentalmente incompatible con el objetivo de proteger la libertad de 
los usuarios de modificar el software. El modelo sistem\'{a}tico de este 
abuso ocurre en el \'{a}mbito de los productos de uso personal, lo cual es 
precisamente donde es m\'{a}s inaceptable. Por consiguiente, hemos dise\~nado 
esta versi\'{o}n de la GPL para prohibir la pr\'{a}ctica de estos productos. Si 
estos problemas surgen de forma substancial en otro dominios, estamos 
preparados para extender esta disposici\'{o}n a esos dominios en futuras 
versiones de la GPL, as\'{i} como sea necesario para proteger la libertad de 
los usuarios.

Por \'{u}ltimo, todo programa es amenazado constantemente por las patentes 
de software. Los Estados no deber\'{i}an permitir patentes que restringen el 
desarrollo y el uso de software en ordenadores de prop\'{o}sito general, 
pero en aquellos que lo hacen, deseamos evitar el peligro particular de 
que las patentes aplicadas a un programa libre podr\'{i}an convertirlo de 
forma efectiva en propietario. Para prevenir esto, la GPL garantiza que 
las patentes no pueden ser utilizadas para hacer que el programa no sea 
libre.

Los t\'{e}rminos exactos y las condiciones para la copia, distribuci\'{o}n y 
modificaci\'{o}n se exponen a continuaci\'{o}n.  
\end{abstract}

\begin{center}
{\Large \sc T\'{e}rminos y Condiciones}
\end{center}


\begin{enumerate}

\addtocounter{enumi}{-1}

\item Definiciones

  ``Esta Licencia" se refiere a la versi\'on 3 de la Licencia P\'ublica General de
GNU.

  ``Derechos de autor ("copyright``)" tambi\'en se refiere a las leyes similares 
a la de derechos de autor (``copyright") que se apliquen a otro tipo de obras, 
tales como las m\'ascaras usadas en la fabricaci\'on de semiconductores.

  ``El Programa" se refiere a cualquier obra con derechos de autor 
(``copyright") bajo esta Licencia. Cada licenciatario es tratado como ``usted". 
Los ``Licenciatarios" y los ``destinatarios" pueden ser individuos u 
organizaciones.

  ``Modificar" una obra quiere decir copiar de ella o adaptar parte o la 
totalidad de la obra de una forma que se requieran permisos de derechos de 
autor (``copyright"), distintos de los de hacer una copia exacta. La obra 
resultante es llamada ``versi\'on modificada" de la obra previa o una obra 
``basada en" la obra previa.

  Una ``obra amparada" significa o el Programa sin modificar o una obra 
basada en el Programa.

  ``Difundir" una obra significa hacer cualquier cosa con ella que, sin
permiso, le har\'ia responsable de forma directa o indirecta de infringir la
ley correspondiente de derechos de autor (``copyright"), excepto ejecutarla 
en un ordenador o modificar una copia privada. La difusi\'on incluye copiar, 
la distribuci\'on (con o sin modificaci\'on), hacerla disponible para el 
p\'ublico, y en algunos pa\'ises tambi\'en otras actividades.

  ``Transmitir" una obra quiere decir cualquier tipo de difusi\'on que permita 
a otras partes hacer o recibir copias. La mera interacci\'on con un usuario a 
trav\'es de una red inform\'atica, sin la transferencia de una copia, no es 
transmitir.

  Una interfaz interactiva de usuario muestra ``Avisos Legales Apropiados" en
la medida que incluye una caracter\'istica visible pr\'actica y destacable que
(1) muestra un aviso apropiado de copyright, e (2) informa al usuario de que
no hay garant\'ia para la obra (excepto las garant\'ias proporcionadas), que los
licenciatarios pueden transmitir la obra bajo esta Licencia, y c\'omo ver una
copia de esta Licencia. Si la interfaz presenta una lista de comandos de
usuario u opciones, como un men\'u, un elemento destacado en la lista
satisface este criterio.

\item C\'odigo Fuente
  
  El ``c\'odigo fuente" de una obra significa la forma preferida de trabajo
para hacerle modificaciones. ``C\'odigo objeto" es cualquier forma no-fuente de
una obra.

  Una ``Interfaz Est\'andar" significa una interfaz que es un est\'andar oficial
definido por un cuerpo de est\'andares reconocido o, en el caso de interfaces
especificadas para un lenguaje de programaci\'on en particular, una que es
extensamente utilizada entre los desarrolladores que trabajan en ese
lenguaje.

  El ``Sistema de Bibliotecas" de una obra ejecutable incluye cualquier cosa,
diferente de la obra como un todo, que (a) est\'a incluido en la forma normal
de paquetizado de un Componente Importante, y (b) sirve solo para habilitar
el uso de la obra con ese Componente Importante, o para implementar una
Interfaz Est\'andar para la cual la implementaci\'on est\'a disponible para el
p\'ublico en forma de c\'odigo fuente. Un ``Componente Importante", en este
contexto, significa un componente esencial importante (kernel, sistema de
ventanas, etc\'etera) del sistema operativo en concreto (si hubiese) en el
cual el ejecutable funciona, o un compilador utilizado para producir la
obra, o un int\'erprete de c\'odigo objeto utilizado para hacerlo funcionar.

  La ``Fuente Correspondiente" de una obra en forma de c\'odigo objeto
significa todo el c\'odigo fuente necesario para generar, instalar, y (para
una obra ejecutable) hacer funcionar el c\'odigo objeto y modificar la obra,
incluyendo scripts para controlar dichas actividades. Sin embargo, ello no
incluye la obra del Sistema de Bibliotecas, o herramientas de prop\'osito
general o programas de libre disponibilidad general los cuales son usados
sin modificaciones para la realizaci\'on de dichas actividades, pero que no
son parte de la obra. Por ejemplo, la Fuente Correspondiente incluye
ficheros de definici\'on de interfaces asociados a los ficheros fuente para la
obra, y el c\'odigo fuente para bibliotecas compartidas y subprogramas enlazados
din\'amicamente para los que la obra est\'a espec\'ificamente dise\~nado para
requerir, tales como comunicaci\'on de datos intr\'inseca o flujo de control entre
aquellos subprogramas y otras partes de la obra.

  La Fuente Correspondiente es necesario que no incluya nada que los
usuarios puedan regenerar autom\'aticamente desde otras partes de la Fuente
Correspondiente.

  La Fuente Correspondiente de una obra en forma de c\'odigo fuente es la obra
en s\'i.

\item Permisos B\'asicos

  Todos los derechos concedidos bajo esta Licencia se conceden durante la 
duraci\'on de los derechos de autor (``copyright'') del Programa, y son 
irrevocables siempre que se cumplan las condiciones establecidas. Esta 
Licencia afirma expl\'icitamente su ilimitado permiso para ejecutar el 
Programa sin modificar. La salida de la ejecuci\'on de una obra amparada est\'a 
amparada por esta Licencia solo si la salida, dado su contenido, constituye 
una obra amparada. Esta Licencia reconoce sus derechos de uso razonable u 
otro equivalente, seg\'un lo establecido por la ley de derechos de autor 
(``copyright'').

  Usted podr\'a realizar, ejecutar y difundir obras amparadas que usted no 
transmita, sin condici\'on alguna, siempre y cuando no tenga otra licencia 
vigente. Podr\'a distribuir obras amparadas a terceros con el \'unico prop\'osito 
de que ellos hagan modificaciones exclusivamente para usted, o proporcionarle 
ayuda para ejecutar estas obras, siempre y cuando cumpla con los t\'erminos 
de esta Licencia en la transmisi\'on de todo el material del cual usted no 
controle los derechos de autor (``copyright''). Aquellos que realicen o 
ejecuten las obras amparadas por usted, deben hacerlo exclusivamente en su 
nombre, bajo su direcci\'on y control, en los t\'erminos que le prohiban 
realizar ninguna copia de su trabajo con derechos de autor (``copyright'') 
fuera de su relaci\'on con usted.

  La distribuci\'on bajo otras circunstancias se permite \'unicamente bajo las
condiciones expuestas a continuaci\'on. No est\'a permitido sublicenciar, la
secci\'on 10 hace que sea innecesario.

\item Protecci\'on de los Derechos Legales de los Usuarios frente a la Ley Antievasi\'on

  Ninguna obra amparada debe considerarse parte de una medida tecnol\'ogica 
efectiva, a tenor de lo establecido en cualquier ley aplicable que cumpla 
las obligaciones expresas en el art\'iculo 11 del tratado de copyright WIPO 
adoptado el 20 de diciembre de 1996, o leyes similares que prohiban o 
restrinjan la evasi\'on de tales medidas.

  Cuando distribuya una obra amparada, renuncia a cualquier poder legal para
prohibir la evasi\'on de medidas tecnol\'ogicas mientras tales evasiones se
realicen en ejercicio de derechos amparados por esta Licencia respecto a la
obra amparada; adem\'as, usted renunciar\'a a cualquier intenci\'on de limitar el
uso o modificaci\'on del trabajo con el objetivo de imponer, contra el trabajo
de los usuarios, sus derechos legales o los de terceros para prohibir la
evasi\'on de medidas tecnol\'ogicas.

\item Distribuci\'on de Copias Literales

  Usted podr\'a distribuir copias literales del c\'odigo fuente del Programa tal
cual lo ha recibido, por cualquier medio, siempre que publique visible y
apropiadamente en cada copia el correspondiente aviso de copyright; mantenga
intactos todos los avisos que establezcan que esta Licencia y cualquier
cl\'ausula no-permisiva a\~nadida acorde con la cl\'ausula 7 son aplicables al
c\'odigo; mantenga intactos todos los avisos de ausencia de garant\'ia; y
proporcione a todos los destinatarios una copia de esta Licencia junto con
el Programa.

  Usted podr\'a cobrar cualquier importe o no cobrar nada por cada copia que
distribuya, y podr\'a ofrecer soporte o protecci\'on de garant\'ia mediante un
pago.

\item Transmisi\'on de Versiones Modificadas de la Fuente

  Usted puede transmitir una obra basada en el Programa, o las
modificaciones para generarla a partir del Programa, en la forma de c\'odigo
fuente bajo los t\'erminos de la secci\'on 4, suponiendo que adem\'as cumpla las
siguientes condiciones:
    \begin{enumerate}
    \item La obra debe incluir avisos destacados indicando que usted la ha 
    modificado y dando una fecha pertinente.

    \item La obra debe incluir avisos destacados indicando que est\'a liberada
    bajo esta Licencia y cualquier otra condici\'on a\~nadida bajo la secci\'on 7.
    Este requerimiento modifica los requerimientos de la secci\'on 4 de
    ``mantener intactos todos los avisos".

    \item Usted debe licenciar la obra entera, como una unidad, bajo esta 
    Licencia para cualquier persona que est\'e en posesi\'on de una copia. Esta
    Licencia se aplicar\'a por consiguiente, junto con cualquier t\'ermino
    aplicable adicional de la secci\'on 7, a la totalidad de la obra, y a
    todos sus componentes, independientemente de como est\'en empaquetados.
    Esta Licencia no da permiso para licenciar la obra de otra forma, pero
    no invalida esos permisos si usted los ha recibido de forma separada.

    \item Si la obra tiene interfaces de usuario interactivas, cada una debe
    mostrar los Avisos Legales Apropiados; sin embargo, si el Programa tiene
    interfaces interactivas que no muestren los Avisos Legales Apropiados,
    tampoco es necesario que su obra lo haga.
    \end{enumerate}
  Una recopilaci\'on de una obra amparada con otras obras separadas e
independientes, que no son por su naturaleza extensiones de la obra
amparada, y que no se combinan con ella con el fin de formar un programa m\'as
grande, en o sobre un volumen de un medio de almacenamiento o distribuci\'on,
es llamado un ``agregado" si la recopilaci\'on y su resultante copyright no son
usados para limitar el acceso o los derechos legales de los usuarios de la
recopilaci\'on m\'as all\'a de lo que las obras individuales permitan. La
inclusi\'on de una obra amparada en un agregado no provoca que esta Licencia
se aplique a los otros componentes del agregado.

\item Transmisi\'on en Forma de No-Fuente

  Usted puede transmitir una obra amparada en forma de c\'odigo objeto bajo
los t\'erminos de las secciones 4 y 5, siempre que tambi\'en transmita la Fuente
Correspondiente legible por una m\'aquina bajo los t\'erminos de esta Licencia,
de una de las siguientes formas:
    \begin{enumerate}
    \item Transmitir el c\'odigo objeto en, o embebido en, un producto f\'isico 
    (incluyendo medios de distribuci\'on f\'isicos), acompa\~nado de la Fuente 
    Correspondiente en un medio f\'isico duradero habitual para el 
    intercambio de software.

    \item Transmitir el c\'odigo objeto en, o embebido en, un producto f\'isico 
    (incluyendo medios de distribuci\'on f\'isicos), acompa\~nado de un 
    ofrecimiento escrito, v\'alido durante al menos tres a\~nos y v\'alido
    mientras usted ofrezca recambios o soporte para ese modelo de producto,
    de dar a cualquiera que posea el c\'odigo objeto o (1) una copia de la
    Fuente Correspondiente de todo el software en el producto amparado por
    esta Licencia, en un medio f\'isico duradero habitual para el intercambio
    de software, por un precio no m\'as elevado que el coste razonable de la
    realizaci\'on f\'isica de la transmisi\'on de la fuente, o (2) acceso para
    copiar la Fuente Correspondiente de un servidor de red sin costo alguno.

    \item Transmitir copias individuales del c\'odigo objeto con una copia del
    ofrecimiento escrito de proveer la Fuente Correspondiente. Esta 
    alternativa est\'a permitida solo ocasionalmente sin fines comerciales, 
    y solo si usted ha recibido el c\'odigo objeto con ese ofrecimiento, de
    acuerdo con la subsecci\'on 6b.

    \item Transmitir el c\'odigo objeto ofreciendo acceso desde un lugar 
    determinado (gratuitamente o mediante pago), y ofrecer acceso 
    equivalente a la Fuente Correspondiente de la misma manera en el mismo
    lugar sin cargo adicional. No es necesario exigir a los destinatarios
    que copien la Fuente Correspondiente junto con el c\'odigo objeto. Si el
    lugar para copiar el c\'odigo objeto es un servidor de red, la Fuente
    Correspondiente puede estar en un servidor diferente (gestionado por
    usted o un tercero) que soporte facilidades de copia equivalentes,
    siempre que mantenga instrucciones claras junto al c\'odigo objeto
    especificando d\'onde encontrar la Fuente Correspondiente.
    Independientemente de qu\'e servidor albergue la Fuente Correspondiente,
    usted seguir\'a estando obligado a asegurar que est\'a disponible durante el
    tiempo que sea necesario para satisfacer estos requisitos.

    \item Transmitir el c\'odigo objeto usando una transmisi\'on peer-to-peer, 
    siempre que informe a los otros usuarios donde se ofrece el c\'odigo 
    objeto y la Fuente Correspondiente de la obra al p\'ublico general de 
    forma gratuita bajo la subsecci\'on 6d.
    \end{enumerate}
  Una porci\'on separable del c\'odigo objeto, cuyo c\'odigo fuente est\'a excluido
de la Fuente Correspondiente, como una Biblioteca del Sistema, no necesita
ser incluida en la distribuci\'on del c\'odigo objeto de la obra.

  Un ``Producto de Usuario" es o (1) un ``producto de consumo", lo que 
significa cualquier propiedad tangible personal que es usada habitualmente
con fines personales, familiares o dom\'esticos, o (2) cualquier cosa dise\~nada
o vendida para ser incorporada en una vivienda. A la hora de determinar
cuando un producto es un producto de consumo, los casos dudosos ser\'an
resueltos en favor de la cobertura. Para un producto concreto recibido por
un usuario concreto, ``uso habitual" se refiere a un uso t\'ipico y com\'un de
esa clase de producto, sin tener en cuenta el estado del usuario concreto o
la forma en la que el usuario concreto realmente use, o espera o se espera
que use, el producto. Un producto es un producto de consumo
independientemente de si el producto tiene usos esencialmente comerciales,
industriales o no comerciales, a menos que dicho uso constituya el \'unico
modo de uso significativo del producto.

  La ``Informaci\'on de Instalaci\'on" de un Producto de Usuario quiere decir
cualquier m\'etodo, procedimiento, clave de autorizaci\'on, u otra informaci\'on
requerida para instalar y ejecutar versiones modificadas de la obra amparada
en ese Producto de Usuario a partir de una versi\'on modificada de su Fuente
Correspondiente. La informaci\'on debe ser suficiente para garantizar que el
funcionamiento continuado del c\'odigo fuente modificado no es prevenido o
interferido por el simple hecho de que ha sido modificado.

  Si usted transmite una obra en c\'odigo objeto bajo esta secci\'on en, o con,
o espec\'ificamete para usar en, un Producto de Usuario, y la transmisi\'on
tiene lugar como parte de una transacci\'on en la cual el derecho de posesi\'on
y uso de un Producto de Usuario es transferido a un destinatario en
perpetuidad o por un periodo establecido (independientemente de c\'omo se
caracterice la operaci\'on), la Fuente Correspondiente transmitida bajo esta
secci\'on debe estar acompa\~nada de la Informaci\'on de Instalaci\'on. Pero este
requisito no se aplica si ni usted ni ning\'un tercero tiene la capacidad de
instalar c\'odigo objeto modificado en el Producto de Usuario (por ejemplo, la
obra ha sido instalada en la ROM).

  El requisito de proveer de la Informaci\'on de Instalaci\'on no incluye el
requisito de continuar proporcionando asistencia, garant\'ia, o
actualizaciones para una obra que ha sido modificada o instalada por el
destinatario, o para un Producto de Usuario en el cual ha sido modificada o
instalada. El acceso a una red puede ser denegado cuando la modificaci\'on en
s\'i afecta materialmente y adversamente el funcionamiento de la red o viola
las reglas y protocolos de comunicaci\'on de la red.

  La Fuente Correspondiente transmitida, y la Informaci\'on de Instalaci\'on 
proporcionada, de acuerdo con esta secci\'on debe estar en un formato que sea
documentado p\'ublicamente (y con una implementaci\'on disponible para el
p\'ublico en formato de c\'odigo fuente), y no deben necesitar contrase\~nas o
claves particulares para la extracci\'on, lectura o copia.

\item T\'erminos Adicionales

  ``Permisos adicionales" son t\'erminos que se a\~naden a los t\'erminos de esta
Licencia haciendo excepciones de una o m\'as de una de sus condiciones. Los
permisos adicionales que son aplicables al Programa entero deber\'an como si
estuvieran incluidos en esta Licencia, en la medida bajo la ley aplicable.
Si los permisos adicionales s\'olo son aplicables a parte del Programa, esa
parte debe ser usada separadamente bajo esos permisos, pero el Programa
completo queda bajo la autoridad de esta Licencia sin considerar los
permisos adicionales.

  Cuando se distribuye una copia de una obra derivada, se puede
opcionalmente quitar cualesqueira permisos adicionales de esa copia, o de
cualquier parte de ella. (Los permisos adiconales deben ser escritos para
requerir su propia eliminaci\'on bajo ciertos casos cuando se modifica la
obra. Se pueden colocar permisos adicionales en material, a\~nadidos a una
obra derivada, para los cu\'ales se establecen o se pueden establecer los
permisos de copyright apropiados.

  No obstante cualquier otra disposici\'on de esta Licencia, para el material
que se a\~nada a una obra derivada, se pueden (si est\'a autorizado por los
titulares del copyright del material) a\~nadir los t\'erminos de esta Licencia
con los siguientes t\'erminos:
    \begin{enumerate}
    \item Ausencia de garant\'ia o limitaci\'on de responsabilidad diferente de 
    los t\'erminos de los art\'iculos 15 y 16 de esta Licencia, o

    \item Exigir la preservaci\'on de determinados avisos legales razonables o
    atribuciones de autor en ese material o en los Avisos Legales Apropiados
    mostrados por los obras que lo contengan, o

    \item Prohibir la tergiversaci\'on del origen de ese material, o requerir que
    las versiones modificadas del material se marquen de maneras razonables
    como diferentes de la versi\'on original, o

    \item Limitar el uso con fines publicitarios de los nombres de los 
    licenciantes o autores del material, o

    \item Negarse a ofrecer derechos afectados por leyes de registro para el
    uso de alguno nombres comerciales, marcas registradas o marcas de
    servicio, o

    \item Exigir la compensaci\'on de los licenciantes y autores de ese material
    por cualquiera que distribuya el material (o versiones modificadas del
    mismo) estableciendo obligaciones contractuales de responsabilidad sobre
    el destinatario, por cualquier responsabilidad que estas obligaciones 
    contractuales impongan directamente sobre los licenciantes y autores.
    \end{enumerate}
  Todos los dem\'as t\'erminos adicionales no-permisivos son consideradas 
``restricciones extra" en el sentido del art\'iculo 10. Si el Programa, tal
cual se recibi\'o, o cualquier parte del mismo, contiene un aviso indicando
que se encuentra cubierto por esta Licencia junto con un t\'ermino que es otra
restricci\'on, se puede quitar ese t\'ermino. Si un documento de licencia
contiene una restricci\'on adicional, pero permite relicenciar o redistribuir
bajo esta Licencia, se puede a\~nadir a un material de la obra derivada bajo
los t\'erminos de ese documento de licencia, a condici\'on de que dicha
restricci\'on no sobreviva el relicenciamiento o redistribuci\'on.

  Si se a\~naden t\'erminos a una obra derivada de acuerdo con esta secci\'on, se
debe colocar, en los archivos fuente involucrados, una declaraci\'on de los
t\'erminos adicionales aplicables a esos archivos, o un aviso indicando d\'onde
encontrar los t\'erminos aplicables.

  Las t\'erminos adicionales, permisivs o no-permisivos, deben aparecer en
forma de una licencia escrita por separado, o figurar como excepciones; los
requisitos anteriores son aplicables en cualquier forma.

\item Conclusiones
  
  Usted no podr\'a difundir o modificar una obra amparada salvo lo expresamente 
permitido por esta Licencia. Cualquier intento diferente de difusi\'on o 
modificaci\'on ser\'a considerado nulo y autom\'aticamente se anular\'an sus derechos 
bajo esta Licencia (incluyendo las licencias de patentes concedidas bajo el 
tercer p\'arrafo de la secci\'on 11).

  Sin embargo, si usted deja de violar esta Licencia, entonces su licencia
desde un poseedor del copyright correspondiente ser\'a restituida (a)
provisionalmente, a menos que y hasta que el titular expl\'icita y finalmente
termine su licencia, y (b) permanentemente, si el titular del copyright no
le ha notificado su violaci\'on por alg\'un medio razonable antes de los 60 d\'ias
siguientes a la cesaci\'on.

  Adem\'as, su licencia desde el poseedor del copyright correspondiente ser\'a
restituida permanentemente si el poseedor del copyrigth le notifica de la
violaci\'on por alg\'un medio razonable, siendo esta la primera vez que recibe
la notificaci\'on de violaci\'on de esta Licencia (para cualquier obra) de ese
titular del copyright, y usted subsana la violaci\'on antes de 30 d\'ias despu\'es
de la recepci\'on de la notificaci\'on.

  La cancelaci\'on de sus derechos bajo esta secci\'on no da por canceladas las
licencias de terceros que hayan recibido copias o derechos de usted bajo
esta Licencia. Si sus derechos han sido cancelados y no fueran renovados de
manera permanente, usted no cumple los requisitos para recibir nuevas
licencias para el mismo material bajo la secci\'on 10.

\item Aceptaci\'on No Obligatoria por Tenencia de Copias
  
  Usted no est\'a obligado a aceptar esta Licencia por recibir o ejecutar una
copia del Programa. La difusi\'on de una obra amparada surgida solamente
como una consecuencia del uso de una transmisi\'on peer-to-peer para recibir
una copia tampoco requiere aceptaci\'on. Sin embargo, esta Licencia solo le
otorga permiso para difundir o modificar cualquier obra amparada. Estas
acciones infringen el copyright si usted no acepta esta Licencia. Por lo
tanto, al modificar o difundir una obra amparada, usted indica que acepta esta
Licencia para poder hacerlo.

\item Herencia Autom\'atica de Licencia para Destinatarios

  Cada vez que transmita una obra amparada, el destinatario recibir\'a 
autom\'aticamente una licencia de los licenciadores originales, para ejecutar, 
modificar y difundir esa obra, sujeto a esa Licencia. Usted no ser\'a 
responsable de asegurar el cumplimiento de esta Licencia por terceros.

  Una ``transacci\'on de entidad" es una transacci\'on que transfiere el control 
de una organizaci\'on, o sustancialmente todos los bienes de una, o subdivide 
una organizaci\'on, o fusiona organizaciones. Si la propagaci\'on de una obra 
amparada surge de una transacci\'on de entidad, cada parte en esa transacci\'on 
que reciba una copia de la obra tambi\'en recibe todas las licencias de la 
obra que la parte interesada tuviese o pudiese ofrecer seg\'un el p\'arrafo 
anterior, adem\'as del derecho a tomar posesi\'on de las Fuentes Correspondientes 
de la obra a trav\'es del predecesor interesado, si el predecesor tiene o 
puede conseguirla con un esfuerzo razonable.

  Usted no podr\'a imponer ninguna restricci\'on posterior en el ejercicio de 
los derechos otorgados o concedidos bajo esta Licencia. Por ejemplo, usted 
no puede imponer un pago por licencia, derechos u otros cargos por el 
ejercicio de los derechos otorgados bajo esta Licencia, y no puede iniciar 
litigios (incluyendo demandas o contrademandas en pleitos) alegando 
cualquier reclamaci\'on de violaci\'on de patentes por cambiar, usar, vender, 
ofrecer en venta o importar el Programa o alguna parte del mismo.

\item Patentes

  Un ``colaborador" es un titular de los derechos de autor (``copyright") que 
autoriza, bajo los t\'erminos de la presente Licencia, el uso del Programa o 
una obra en la que se base el Programa. La obra as\'i licenciada se denomina 
``versi\'on en colaboraci\'on" del colaborador.

  Las ``demandas de patente esenciales" del colaborador son todas las 
reivindicaciones de patentes pose\'idas o controladas por el colaborador, ya 
se encuentren adquiridas o hayan sido adquiridas con posterioridad, que sean 
infringidas de alguna manera, permitidas por esta Licencia, al hacer, usar 
o vender la versi\'on en colaboraci\'on, pero sin incluir demandas que solo sean 
infringidas como consecuencia de modificaciones posteriores de la versi\'on 
en colaboraci\'on. Para los prop\'ositos de esta definici\'on, ``control" incluye 
el derecho de conceder sublicencias de patente de forma consistente con los 
requisitos establecidos en la presente Licencia.

  Cada colaborador le concede una licencia de la patente no-exclusiva, 
global y libre de regal\'ias bajo las demandas de patente esenciales del 
colaborador, para hacer, usar, modificar, vender, ofrecer para venta, 
importar y otras formas de ejecuci\'on, modificaci\'on y difusi\'on del 
contenido de la versi\'on en colaboraci\'on.

  En los siguientes tres p\'arrafos, una ``licencia de patente" se define como 
cualquier acuerdo o compromiso expreso, cualquiera que sea su denominaci\'on, 
que no imponga una patente (como el permiso expreso para ejecutar una 
patente o acuerdos para no imponer demandas por infracci\'on de patente). 
``Conceder" una licencias de patente de este tipo a un tercero significa 
hacer tal tipo de acuerdo o compromiso que no imponga una patente al tercero.

  Si usted transmite una obra amparada, conociendo que est\'a amparada por 
una licencia de patente, y las Fuentes Correspondientes no se encuentran 
disponibles de forma p\'ublica para su copia, sin cargo alguno y bajo los 
t\'erminos de esta Licencia, ya sea a trav\'es de un servidor p\'ublico o mediante 
cualquier otro medio, entonces usted deber\'a (1) hacer que las Fuentes 
Correspondientes sean p\'ublicas, o (2) tratar de eliminar los beneficios de 
la licencia de patente para esta obra en particular, o (3) tratar de 
extender, de manera compatible con los requisitos de esta Licencia, la 
licencia de patente a terceros. ``Conocer que est\'a afectado" significa que 
usted tiene conocimiento real de que, para la licencia de patente, la 
distribuci\'on de la obra amparada en un pa\'is, o el uso de la obra amparada 
por sus destinatarios en un pa\'is, infringir\'ia una o m\'as patentes existentes 
en ese pa\'is que usted considera v\'alidas por alg\'un motivo.

  Si en virtud de o en conexi\'on con alguna transacci\'on o acuerdo, usted 
transmite, o difunde con fines de distribuci\'on, una obra amparada, y 
concede una licencia de patente para alg\'un tercero que reciba la obra 
amparada, y les autorice a usar, transmitir, modificar o difundir una copia 
espec\'ifica de la obra amparada, entonces la licencia de patente que usted 
otorgue se extiende autom\'aticamente a todos los receptores de la obra 
amparada y cualquier obra basada en ella.

  Una licencia de patente es ``discriminatoria" si no incluye dentro de su 
\'ambito de cobertura, prohibe el ejercicio de, o est\'a condicionada a no 
ejercitar uno o m\'as de los derechos que est\'an espec\'ificamente otorgados por 
esta Licencia. Usted no debe transmitir una obra amparada si est\'a implicado 
en un acuerdo con terceros que est\'e relacionado con el negocio de la 
distribuci\'on de software, en el que usted haga pagos a terceros relacionados 
con su actividad de distribuci\'on de la obra, bajo el que terceros conceden, 
a cualquier receptor de la obra amparada, una licencia de patente 
discriminatoria (a) en relaci\'on con las copias de la obra amparada 
transmitidas por usted (o copias hechas a partir de estas), o 
(b) principalmente para y en relaci\'on con productos espec\'ificos o 
compilaciones que contengan la obra amparada, a menos que usted forme parte 
del acuerdo, o que esa licencia de patente fuese concedida antes del 28 de 
marzo de 2007.

  Ninguna cl\'ausula de esta Licencia debe ser considerada como excluyente 
o limitante de cualquier otra licencia implicada u otras defensas legales 
a que pudiera tener derecho bajo la ley de propiedad intelectual vigente.

\item No Abandonar la Libertad de Otros

  Si se le imponen condiciones (bien sea por orden judicial, acuerdo o de 
otra manera) que contradicen las condiciones de esta Licencia, estas no le 
eximen de las condiciones de esta Licencia. Si usted no puede transmitir 
una obra amparada de forma que pueda satisfacer simult\'aneamente sus 
obligaciones bajo esta Licencia y cualesquiera otras obligaciones 
pertinentes, entonces, como consecuencia, usted no puede transmitirla. Por 
ejemplo, si usted est\'a de acuerdo con los t\'erminos que le obligan a cobrar 
una regal\'ia por la transmisi\'on a aquellos a los que transmite el Programa, 
la \'unica forma en la que usted podr\'ia satisfacer tanto esos t\'erminos como 
esta Licencia ser\'ia abstenerse completamente de transmitir el Programa.

\item Utilizaci\'on con la Licencia P\'ublica General Affero de GNU

  A pesar de cualquier otra disposici\'on de esta Licencia, usted tiene
permiso para enlazar o combinar cualquier obra amparada con una obra 
licenciada bajo la Licencia P\'ublica General Affero de GNU en una \'unica obra 
combinada, y para transmitir la obra resultante. Los t\'erminos de esta 
Licencia continuar\'an aplic\'andose a la parte que es la obra amparada, pero 
los requisitos particulares de la Licencia P\'ublica General Affero de GNU, 
secci\'on 13, concernientes a la interacci\'on a trav\'es de una red se aplicar\'an 
a la combinaci\'on como tal.

\item Versiones Revisadas de esta Licencia

  La Free Software Foundation puede publicar versiones revisadas y/o nuevas 
de la Licencia General P\'ublica de GNU de vez en cuando. Cada nueva versi\'on 
ser\'a similar en esp\'iritu a la versi\'on actual, pero puede diferir en 
detalles para abordar nuevos problemas o preocupaciones.

  Cada versi\'on recibe un n\'umero de versi\'on distintivo. Si el Programa 
especifica que cierta versi\'on numerada de la Licencia General P\'ublica de 
GNU ``o cualquier versi\'on posterior" se aplica a \'el, usted tiene la opci\'on 
de seguir los t\'erminos y condiciones de esa versi\'on numerada o de cualquier 
versi\'on posterior publicada por la Free Software Foundation. Si el Programa 
no especifica un n\'umero de versi\'on de la Licencia General P\'ublica de GNU, 
usted puede escoger cualquier versi\'on publicada por la Free Software 
Foundation.

  Si el Programa escifica que un representante puede decidir que versiones 
futuras de la Licencia General P\'ublica de GNU pueden ser utilizadas, la 
declaraci\'on p\'ublica del representante de aceptar una versi\'on permanentemente 
le autoriza a usted a elegir esa versi\'on para el Programa.

  Las versiones posteriores de la licencia pueden darle permisos adicionales 
o diferentes. No obstante, no se impone a ning\'un autor o titular de los 
derechos de autor obligaciones adicionales como resultado de su elecci\'on de 
seguir una versi\'on posterior.

\item Descargo de Responsabilidad de Garant\'ia

  NO HAY GARANT\'IA PARA EL PROGRAMA, PARA LA EXTENSI\'ON PERMITIDA POR LA LEY 
APLICABLE. EXCEPTO CUANDO SE INDIQUE LO CONTRARIO POR ESCRITO, LOS TITULARES 
DE LOS DERECHOS DE AUTOR (``COPYRIGHT") Y/O TERCEROS PROPORCIONAN EL PROGRAMA 
``TAL CUAL" SIN GARANT\'IAS DE NING\'UN TIPO, BIEN SEAN EXPL\'ICITAS O IMPL\'ICITAS, 
INCLUYENDO, PERO NO LIMITADO A, LAS GARANT\'IAS IMPL\'ICITAS DE COMERCIALIZACI\'ON 
Y APTITUD PARA UN PROP\'OSITO PARTICULAR. EL RIESGO TOTAL EN CUANTO A CALIDAD 
Y RENDIMIENTO DEL PROGRAMA ES CON USTED. SI EL PROGRAMA PRESENTA ALG\'UN 
DEFECTO, USTED ASUME EL COSTO DE TODAS LAS REVISIONES NECESARIAS, 
REPARACIONES O CORRECCIONES.

\item Limitaci\'on de la Responsabilidad

En ning\'un caso, expcepto si se es requerido por la aplicaci\'on de alguna ley 
o alguna aceptaci\'on escrita, podr\'a el poseedor del Copyright, o alg\'un 
tercero que modifique y/o comunique el programa como se permite arriba, 
podr\'a usted reclamar su responsabilidad por da\~nos, incluyendo da\~nos 
generales, especiales, incidentales o da\~nos consecuencia del uso o 
prohibici\'on de uso del programa (incluyendo, pero no limit\'andose a: la 
p\'erdida de informaci\'on o informaci\'on interpretada incorrectamente, o 
p\'erdidas producidas por usted o terceras personas, o fallos del programa al 
ser operado junto a otros programas), incluso si dicho poseedor u otro 
tercero ha sido avisado de la posibilidad de dicho da\~no.

17. Interpretaci\'on de las Secciones 15 y 16.

Si el descargo de responsabilidad de garant\'ia y el l\'imite de responsabilidad 
proporcionado anteriormente no tiene efectos legales de acuerdo a sus 
t\'erminos, el tribunal de revisi\'on debe aplicar la ley local que m\'as se asemeje
a una renuncia absoluta de la responsabilidad civil concerniente al Programa,
a menos que una garant\'ia o una asunci\'on de responsabilidad acompa\~ne a la copia
del Programa junto a un pago de honorarios.

\begin{center}
{\Large\sc Fin de los t\'{e}rminos y condiciones}

\bigskip
C\'{o}mo Aplicar Estos T\'{e}rminos a Sus Nuevos Programas
\end{center}

Si desarrolla un nuevo programa, y quiere que sea lo m\'{a}s usado posible por
el p\'{u}blico, la mejor manera de conseguirlo es hacerlo software libre para que
cualquiera pueda redistribuirlo y modificarlo bajo estos t\'{e}rminos.

Para ello, a\~nada la siguiente nota al programa. Lo m\'{a}s seguro es a\~nadirla
al principio de cada fichero fuente para declarar m\'{a}s efectivamente la
exclusi\'{o}n de garant\'{i}a; y cada fichero debe tener al menos la l\'{i}nea de ``derechos de autor (``copyright'')''
y un puntero a donde se pueda encontrar la anotaci\'{o}n completa.

{\footnotesize
\begin{verbatim}
<una l�nea para dar el nombre del programa y una breve idea de lo que hace>
Copyright (C) <a�o>  <nombre del autor>
    
Este programa es software libre: puede redistribuirlo y/o modificarlo bajo
los t�rminos de la Licencia General P�blica de GNU publicada por la Free
Software Foundation, ya sea la versi�n 3 de la Licencia, o (a su elecci�n)
cualquier versi�n posterior.
    
Este programa se distribuye con la esperanza de que sea �til pero SIN
NINGUNA GARANT�A; incluso sin la garant�a impl�cita de MERCANTIBILIDAD o
CALIFICADA PARA UN PROP�SITO EN PARTICULAR. Vea la Licencia General P�blica 
de GNU para m�s detalles.
    
Usted ha debido de recibir una copia de la Licencia General P�blica 
de GNU junto con este programa. Si no, vea <http://www.gnu.org/licenses/>.
\end{verbatim}
}

Tambi\'{e}n a\~nada informaci\'{o}n sobre c\'{o}mo contactarle por correo electr\'{o}nico u
ordinario.

Si el programa es interactivo, haga que muestre un breve aviso como el 
siguiente cuando se inicie en modo interactivo:

{\footnotesize
\begin{verbatim}
<programa>  Copyright (C) <a�o>  <nombre del autor>
Este programa se ofrece SIN GARANT�A ALGUNA; escriba `show w' para 
consultar los detalles. Este programa es software libre, y usted puede 
redistribuirlo bajo ciertas condiciones; escriba `show c' para mas 
informacion.
\end{verbatim}
}

Los hipot\'{e}ticos comandos {\tt show w} y {\tt show w} deber\'{a}n mostrar las partes 
correspondientes de la Licencia General P\'{u}blica. Por supuesto, los comandos 
en su programa pueden ser diferentes; para una interfaz gr\'{a}fica de usuario, 
puede usar un mensaje del tipo ``Acerca de''.

Tambi\'{e}n deber\'{i}a conseguir que su empresa (si trabaja como programador) o 
escuela, en su caso, firme una ``renuncia de derechos de autor (``copyright'')'' sobre el programa, si 
fuese necesario. Para m\'{a}s informaci\'{o}n a este respecto, y saber c\'{o}mo aplicar 
y cumplir la licencia GNU GPL, consulte \texttt{http://www.gnu.org/licenses/}. 

La Licencia General P\'{u}blica de GNU no permite incorporar sus programas como 
parte de programas propietarios. Si su programa es una subrutina en una 
biblioteca, podr\'{i}a considerar mucho m\'{a}s \'{u}til permitir el enlace de 
aplicaciones propietarias con la biblioteca. Si esto es lo que quiere hacer, 
utilice la GNU Lesser General Public License en vez de esta Licencia. Pero primero, por favor consulte

\texttt{http://www.gnu.org/philosophy/why-not-lgpl.html}.

\end{enumerate}

\end{document}
