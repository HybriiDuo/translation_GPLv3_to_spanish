\documentclass[11pt]{article}
\usepackage[spanish]{babel}

\title{LICENCIA P\'{U}BLICA GENERAL DE GNU}
\date{Versi\'{o}n 3, 29 de junio de 2007}

\begin{document}
\maketitle

\textit{Esta es una traducci\'{o}n no oficial al espa\~nol de la GNU General Public License. No ha sido publicada por la Free Software Foundation, y no establece legalmente las condiciones de distribuci\'{o}n para el software que usa la GNU GPL --estas condiciones se establecen solamente por el texto original, en ingl\'{e}s, de la GNU GPL. Sin embargo, esperamos que esta traducci\'{o}n ayude a los hispanohablantes a entender mejor la GNU GPL. }

\vspace{0.4cm}

\textit{This is an unofficial translation of the GNU General Public License 
into Spanish. It was not published by the Free Software Foundation, and 
does not legally state the distribution terms for software that uses the 
GNU GPL --only the original English text of the GNU GPL does that. However, 
we hope that this translation will help Spanish speakers understand the 
GNU GPL better.}

\begin{center}
{\parindent 0in

Copyright \copyright\  2007 Free Software Foundation, Inc. \texttt{http://fsf.org/}
  
 
\bigskip
Se permite la copia y distribuci\'{o}n de copias literales de este documento, pero no se permite su modificaci\'{o}n. }

\end{center}

\renewcommand{\abstractname}{Pre\'{a}mbulo}
\begin{abstract}
La Licencia P\'{u}blica General de GNU es una licencia libre, bajo ``copyleft'', para software y otro tipo de obras.

Las licencias para la mayor\'{i}a del software y otras obras de 
car\'{a}cter pr\'{a}ctico est\'{a}n dise\~nadas para privarle de la libertad 
de compartir y modificar las obras. Por el contrario, la Licencia P\'{u}blica General de GNU pretende garantizar su libertad de compartir y modificar 
todas las versiones de un programa --para cerciorar que permanece como 
software libre para todos sus usuarios. Nosotros, la Free Software Foundation, usamos la Licencia P\'{u}blica General de GNU para la mayor\'{i}a 
de nuestro software; la cual se aplica tambi\'{e}n a cualquier otra 
obra publicada de esta forma por parte de sus autores. Usted tambi\'{e}n 
puede aplicarla a sus programas.

Cuando hablamos de software libre (free software), nos referimos a libertad, no a precio. Nuestras Licencias P\'{u}blicas Generales est\'{a}n dise\~nadas para garantizar su libertad de distribuir copias de software libre (y cobrar por ellas si lo desea), recibir el c\'{o}digo fuente o poder obtenerlo si quiere, modificar el software o usar fragmentos de \'{e}l en sus nuevos programas, y que sepa que puede hacer esas cosas.

Para proteger sus derechos, necesitamos impedir que otros le denieguen 
esos derechos o que le pidan que renuncie a ellos. Por ello, tiene 
ciertas responsabilidades si distribuye copias del software, o si lo 
modifica: la responsabilidad de respetar la libertad de otros.

Por ejemplo, si distribuye copias de un programa, bien sea gratis o 
por una tasa, debe transferirles a los que lo reciban las mismas 
libertades que usted recibi\'{o}.  Debe asegurarse que ellos, tambi\'{e}n, 
reciben o pueden obtener el c\'{o}digo fuente. Y debe mostrarles estos 
t\'{e}rminos para que ellos puedan conocer sus derechos.

Los desarrolladores que usan la GNU GPL protegen tus derechos con dos 
pasos:
(1) haciendo valer el derecho de propiedad intelectual en el software, y 
(2) ofreci\'{e}ndole esta Licencia que le da el permiso legal para copiarlo, 
distribuirlo y/o modificarlo.

Para la protecci\'{o}n de autores y desarrolladores, la GPL explica 
claramente que no hay garant\'{i}a para este software libre. Por el bien 
tanto de usuarios como de autores, la GPL requiere que las versiones 
modificadas sean marcadas como con cambios, de forma que sus problemas 
no puedan ser atribuidos de forma err\'{o}nea a autores de versiones previas.

Algunos dispositivos est\'{a}n dise\~nados para denegar a los usuarios el 
acceso para instalar o ejecutar versiones modificadas del software en su 
interior, a pesar de que el fabricante puede hacerlo. Esto es 
fundamentalmente incompatible con el objetivo de proteger la libertad de 
los usuarios de modificar el software. El modelo sistem\'{a}tico de este 
abuso ocurre en el \'{a}mbito de los productos de uso personal, lo cual es 
precisamente donde es m\'{a}s inaceptable. Por consiguiente, hemos dise\~nado 
esta versi\'{o}n de la GPL para prohibir la pr\'{a}ctica de estos productos. Si 
estos problemas surgen de forma substancial en otro dominios, estamos 
preparados para extender esta disposici\'{o}n a esos dominios en futuras 
versiones de la GPL, as\'{i} como sea necesario para proteger la libertad de 
los usuarios.

Por \'{u}ltimo, todo programa es amenazado constantemente por las patentes 
de software. Los Estados no deber\'{i}an permitir patentes que restringen el 
desarrollo y el uso de software en ordenadores de prop\'{o}sito general, 
pero en aquellos que lo hacen, deseamos evitar el peligro particular de 
que las patentes aplicadas a un programa libre podr\'{i}an convertirlo de 
forma efectiva en propietario. Para prevenir esto, la GPL garantiza que 
las patentes no pueden ser utilizadas para hacer que el programa no sea 
libre.

Los t\'{e}rminos exactos y las condiciones para la copia, distribuci\'{o}n y 
modificaci\'{o}n se exponen a continuaci\'{o}n.  
\end{abstract}

\begin{center}
{\Large \sc T\'{e}rminos y Condiciones}
\end{center}


\begin{enumerate}

\addtocounter{enumi}{-1}

\item Definiciones.

``Esta Licencia'' se refiere a la versi\'{o}n 3 a la Licencia P\'{u}blica 
General de GNU.

``Derechos de Autor (``Copyright'')'' tambi\'{e}n significa las leyes similares a la de derechos de autor (``copyright'') 
que se apliquen a otro tipo de obras, tales como las mascaras usadas 
en la fabricacion de semiconductores.

``El Programa'' se refiere a cualquier obra con derechos de autor (``copyright'') 
bajo esta Licencia.  Cada licenciatario es tratado como ``usted''. 
Los ``Licenciatarios'' y los ``destinatarios'' pueden ser individuos u organizaciones.

``Modificar'' una obra quiere decir copiar de ella o adaptar parte o la 
totalidad de la obra de una forma que se requieran permisos de derechos de autor (``copyright''), 
distintos de los de hacer una copia exacta.  La obra resultante es 
llamada ``versi\'{o}n modificada'' de la obra previa o una obra ``basada en'' 
la obra previa.

Una ``obra amparada'' significa o el Programa sin modificar o una obra 
basada en el Programa.

``Propagar'' una obra significa hacer cualquier cosa con ella que, sin
permiso, le har\'{i}a responsable de forma directa o indirecta de infringir
la ley correspondiente de derechos de autor (``copyright''), excepto ejecutarla en un ordenador
o modificar una copia privada.  La propagaci\'{o}n incluye copiar, la 
distribuci\'{o}n (con o sin modificaci\'{o}n), hacerla disponible para el 
p\'{u}blico, y en algunos pa\'{i}ses tambi\'{e}n otras actividades.

``Transmitir'' una obra quiere decir cualquier tipo de propagaci\'{o}n que 
permita a otras partes hacer o recibir copias.  La mera interacci\'{o}n con
un usuario a trav\'{e}s de una red inform\'{a}tica, sin la transferencia de una 
copia, no es transmitir.

Una interfaz interactiva de usuario muestra ``Avisos Legales Apropiados''
en la medida que incluye una caracter\'{i}stica visible pr\'{a}ctica y 
destacable que (1) muestra un aviso apropiado de derechos de autor (``copyright''), e (2)
informa al usuario de que no hay garant\'{i}a para la obra (excepto las 
garant\'{i}as proporcionadas), que los licenciatarios pueden transmitir la 
obra bajo esta Licencia, y c\'{o}mo ver una copia de esta Licencia. 
Si la interfaz presenta una lista de comandos de usuario u opciones, 
como un men\'{u}, un elemento destacado en la lista satisface este criterio.

\item C\'{o}digo Fuente.

El ``c\'{o}digo fuente'' de una obra significa la forma preferida de
trabajo para hacerle modificaciones. ``C\'{o}digo objeto'' es cualquier 
forma no-fuente de una obra.

Una ``Interfaz Est\'{a}ndar'' significa una interfaz que es un est\'{a}ndar 
oficial definido por un cuerpo de est\'{a}ndares reconocido o, en el 
caso de interfaces especificadas para un lenguaje de programaci\'{o}n en
particular, una que es extensamente utilizada entre los
desarrolladores que trabajan en ese lenguaje.

Las ``Bibliotecas del Sistema'' de una obra ejecutable incluyen
cualquier cosa, diferente de la obra como un todo, que (a) est\'{a}n incluidas en la forma normal de paquetizado de un Componente Importante, y (b) sirve solo para habilitar el uso de la obra con ese Componente Importante, o para implementar una Interfaz Est\'{a}ndar
para la cual la implementaci\'{o}n est\'{a} disponible para el p\'{u}blico en
forma de c\'{o}digo fuente. Un ``Componente Importante'', en este
contexto, significa un componente esencial importante (kernel,
sistema de ventanas, etc\'{e}tera) del sistema operativo en concreto (si hubiese) en el cual el ejecutable funciona, o un compilador
utilizado para producir la obra, o un int\'{e}rprete de c\'{o}digo objeto
utilizado para hacerlo funcionar.

La ``Fuente Correspondiente'' de una obra en forma de c\'{o}digo
objeto significa todo el c\'{o}digo fuente necesario para generar,
instalar, y (para una obra ejecutable) hacer funcionar el c\'{o}digo
objeto y modificar la obra, incluyendo scripts para controlar
dichas actividades. Sin embargo, ello no incluye la obra de las Bibliotecas del Sistema, o herramientas de prop\'{o}sito general o
programas de libre disponibilidad general los cuales son usados sin
modificaciones para la realizaci\'{o}n de dichas actividades, pero que no son parte de la obra. Por ejemplo, la Fuente Correspondiente incluye ficheros de definici\'{o}n de interfaces asociados con los ficheros fuente para la obra, y el c\'{o}digo fuente para bibliotecas compartidas y subprogramas enlazados din\'{a}micamente para los que la obra est\'{a} espec\'{i}ficamente dise\~nado para requerir, tales como comunicaci\'{o}n de datos intr\'{i}nseca o flujo de control entre aquellos subprogramas y otras partes de la obra.

La Fuente Correspondiente es necesario que no incluya nada que los usuarios puedan regenerar autom\'{a}ticamente desde otras partes de la Fuente Correspondiente.

La Fuente Correspondiente de una obra en forma de c\'{o}digo fuente es la obra en s\'{i}.

\item Permisos b\'{a}sicos.

Todos los derechos concedidos bajo esta Licencia se conceden durante la duraci\'{o}n del derechos de autor (``copyright'') del Programa, y son irrevocables siempre que se cumplan las condiciones establecidas. Esta Licencia afirma expl\'{i}citamente su ilimitado permiso para ejecutar el Programa sin modificar. La salida de la ejecuci\'{o}n de un programa amparado est\'{a} amparada por esta Licencia solo si la salida, dado su contenido, constituye una obra amparada. Esta Licencia reconoce sus derechos de uso razonable u otro equivalente, seg\'{u}n lo establecido por la ley de derechos de autor (``copyright'').

Usted podr\'{a} realizar, ejecutar y difundir obras amparadas que usted no transmita, sin condici\'{o}n alguna, siempre y cuando no tenga otra licencia vigente. Podr\'{a} distribuir obras amparadas a terceros con el \'{u}nico prop\'{o}sito de que ellos hagan modificaciones exclusivamente para usted, o proporcionarle ayuda para ejecutar estas obras, siempre y cuando cumpla con los t\'{e}rminos de esta Licencia en la transmisi\'{o}n de todo el material del cual usted no controle el derechos de autor (``copyright''). Aquellos que realicen o ejecuten las obras amparadas por usted, deben hacerlo exclusivamente en su nombre, bajo su direcci\'{o}n y control, en los t\'{e}rminos que le prohiban realizar ninguna copia de su trabajo con derechos de autor (``copyright'') fuera de su relaci\'{o}n con usted.

La distribuci\'{o}n bajo otras circunstancias se permite \'{u}nicamente bajo las condiciones expuestas a  continuaci\'{o}n. No est\'{a} permitido sublicenciar, la secci\'{o}n 10 hace que sea innecesario.

\item Protecci\'{o}n de los Derechos Legales de los Usuarios frente a la 
Ley Antievasi\'{o}n.

Ninguna obra amparada debe considerarse parte de una medida tecnol\'{o}gica 
efectiva, a tenor de lo establecido en cualquier ley aplicable que cumpla 
las obligaciones expresas en el art\'{i}culo 11 del tratado de derechos de autor (``copyright'') de WIPO 
adoptado el 20 de diciembre de 1996, o leyes similares que prohiban o 
restrinjan la evasi\'{o}n de tales medidas.

Cuando transmita una obra amparada, renuncia a cualquier poder legal 
para prohibir la evasi\'{o}n de medidas tecnol\'{o}gicas mientras tales evasiones 
se realicen en ejercicio de derechos amparados por esta Licencia respecto 
a la obra amparada; adem\'{a}s, usted renunciar\'{a} a cualquier intenci\'{o}n de 
limitar el uso o modificaci\'{o}n del trabajo con el objetivo de imponer, 
contra el trabajo de los usuarios, sus derechos legales o los de terceros 
para prohibir la evasi\'{o}n de medidas tecnol\'{o}gicas.

\item Transmisi\'{o}n de copias literales.

Usted podr\'{a} distribuir copias literales del c\'{o}digo fuente del 
Programa tal cual lo ha recibido, por cualquier medio, siempre que 
publique visible y apropiadamente en cada copia el correspondiente 
aviso de derechos de autor (``copyright''); mantenga intactos todos los avisos que establezcan 
que esta Licencia y cualquier cl\'{a}usula no-permisiva a\~nadida acorde con 
la cl\'{a}usula 7 son aplicables al c\'{o}digo; mantenga intactos todos los 
avisos de ausencia de garant\'{i}a; y proporcione a todos los destinatarios 
una copia de esta Licencia junto con el Programa.

Usted podr\'{a} cobrar cualquier importe o no cobrar nada por cada copia 
que distribuya, y podr\'{a} ofrecer soporte o protecci\'{o}n de garant\'{i}a 
mediante un pago.

\item Transmisi\'{o}n de Versiones Modificadas de la Fuente.

Usted puede transmitir una obra basada en el Programa, o las 
modificaciones para generarla a partir del Programa, en la forma de 
c\'{o}digo fuente bajo los t\'{e}rminos de la secci\'{o}n 4, suponiendo que adem\'{a}s 
cumpla las siguientes condiciones:

  \begin{enumerate}
  \item La obra debe incluir avisos destacados indicando que usted la ha 
  modificado y dando una fecha pertinente.

  \item La obra debe incluir avisos destacados indicando que est\'{a} 
  liberada bajo esta Licencia y cualquier otra condici\'{o}n a\~nadida bajo 
  la secci\'{o}n 7.  Este requerimiento modifica los requerimientos de la 
  secci\'{o}n 4 de ``mantener intactos todos los avisos''.

  \item Usted debe licenciar la obra entera, como una unidad, bajo esta 
  Licencia para cualquier persona que est\'{e} en posesi\'{o}n de una copia.  
  Esta Licencia se aplicar\'{a} por consiguiente, junto con cualquier 
  t\'{e}rmino aplicable adicional de la secci\'{o}n 7, a la totalidad de la 
  obra, y a todos sus componentes, independientemente de como est\'{e}n 
  empaquetados. Esta Licencia no da permiso para licenciar la obra de 
  otra forma, pero no invalida esos permisos si usted los ha recibido 
  de forma separada.

  \item Si la obra tiene interfaces de usuario interactivas, cada una 
  debe mostrar los Avisos Legales Apropiados; sin embargo, si el Programa 
  tiene interfaces interactivas que no muestren los Avisos Legales 
  Apropiados, tampoco es necesario que su obra lo haga.
\end{enumerate}
Una recopilaci\'{o}n de una obra amparada con otras obras separadas e 
independientes, que no son por su naturaleza extensiones de la obra 
amparada, y que no se combinan con ella con el fin de formar un programa 
m\'{a}s grande, en o sobre un volumen de un medio de almacenamiento o 
distribuci\'{o}n, es llamado un ``agregado'' si la recopilaci\'{o}n y su 
resultante derechos de autor (``copyright'') no son usados para limitar el acceso o los 
derechos legales de los usuarios de la recopilaci\'{o}n m\'{a}s all\'{a} de lo que 
las obras individuales permitan.  La inclusi\'{o}n de una obra amparada en 
un agregado no provoca que esta Licencia se aplique a los otros 
componentes del agregado.

\item Transmisi\'{o}n en Forma de No-Fuente.

Usted puede transmitir una obra amparada en forma de c\'{o}digo objeto 
bajo los t\'{e}rminos de las secciones 4 y 5, siempre que tambi\'{e}n transmita
la Fuente Correspondiente legible por una m\'{a}quina bajo los t\'{e}rminos de 
esta Licencia, de una de las siguientes formas:
  \begin{enumerate}
  \item Transmitir el c\'{o}digo objeto en, o embebido en, un producto f\'{i}sico 
    (incluyendo medios de distribuci\'{o}n f\'{i}sicos), acompa\~nado de la Fuente 
    Correspondiente en un medio f\'{i}sico duradero habitual para el 
    intercambio de software.

  \item Transmitir el c\'{o}digo objeto en, o embebido en, un producto f\'{i}sico 
    (incluyendo medios de distribuci\'{o}n f\'{i}sicos), acompa\~nado de una 
    ofrecimiento escrito, v\'{a}lido durante al menos tres a\~nos y v\'{a}lido 
    mientras usted ofrezca recambios o soporte para ese modelo de 
    producto, de dar a cualquiera que posea el c\'{a}digo objeto o (1) una
    copia de la Fuente Correspondiente de todo el software en el 
    producto amparado por esta Licencia, en un medio f\'{i}sico duradero 
    habitual para el intercambio de software, por un precio no m\'{a}s 
    elevado que el coste razonable de la realizaci\'{o}n f\'{i}sica de la 
    transmisi\'{o}n de la fuente, o (2) acceso para copiar la Fuente 
    Correspondiente de un servidor de red sin costo alguno.

  \item Transmitir copias individuales del c\'{o}digo objeto con una copia 
    del ofrecimiento escrito de proveer la Fuente Correspondiente.  Esta 
    alternativa est\'{a} permitida solo ocasionalmente sin fines comerciales, 
    y solo si usted ha recibido el c\'{o}digo objeto con ese ofrecimiento, 
    de acuerdo con la subsecci\'{o}n 6b.

  \item Transmitir el c\'{o}digo objeto ofreciendo acceso desde un lugar 
    determinado (gratuitamente o mediante pago), y ofrecer acceso 
    equivalente a la Fuente Correspondiente de la misma manera en el 
    mismo lugar sin cargo adicional. No es necesario exigir a los 
    destinatarios que copien la Fuente Correspondiente junto con el 
    c\'{o}digo objeto.  Si el lugar para copiar el c\'{o}digo objeto es un 
    servidor de red, la Fuente Correspondiente puede estar en un 
    servidor diferente (gestionado por usted o un tercero) que soporte 
    facilidades de copia equivalentes, siempre que mantenga 
    instrucciones claras junto al c\'{o}digo objeto especificando d\'{o}nde 
    encontrar la Fuente Correspondiente. Independientemente de qu\'{e} 
    servidor albergue la Fuente Correspondiente, usted seguir\'{a} estando 
    obligado a asegurar que est\'{a} disponible durante el tiempo que 
    sea necesario para satisfacer estos requisitos.

  \item Transmitir el c\'{o}digo objeto usando una transmisi\'{o}n peer-to-peer, 
    siempre que informe a los otros usuarios donde se ofrece el c\'{o}digo 
    objeto y la Fuente Correspondiente de la obra al p\'{u}blico general de 
    forma gratuita bajo la subsecci\'{o}n 6d.
  \end{enumerate}

Una porci\'{o}n separable del c\'{o}digo objeto, cuyo c\'{o}digo fuente est\'{a} 
excluido de la Fuente Correspondiente, como una Biblioteca del Sistema, 
no necesita ser incluida en la distribuci\'{o}n del c\'{o}digo objeto de la obra.

Un ``Producto de Usuario'' es o (1) un "producto de consumo", lo que 
significa cualquier propiedad tangible personal que es usada 
habitualmente con fines personales, familiares o dom\'{e}sticos, o (2) 
cualquier cosa dise\~nada o vendida para ser incorporada en una vivienda.  
A la hora de determinar cuando un producto es un producto de consumo, 
los casos dudosos ser\'{a}n resueltos en favor de la cobertura.  Para un 
producto concreto recibido por un usuario concreto, ``uso habitual'' se 
refiere a un uso t\'{i}pico y com\'{u}n de esa clase de producto, sin tener en 
cuenta el estado del usuario concreto o la forma en la que el usuario 
concreto realmente use, o espera o se espera que use, el producto.  Un 
producto es un producto de consumo independientemente de si el producto 
tiene usos esencialmente comerciales, industriales o no comerciales, a 
menos que dicho uso constituya el \'{u}nico modo de uso significativo del 
producto.

La ``Informaci\'{o}n de Instalaci\'{o}n'' de un Producto de Usuario quiere decir
cualquier m\'{e}todo, procedimiento, clave de autorizaci\'{o}n, u otra 
informaci\'{o}n requerida para instalar y ejecutar versiones modificadas de 
la obra amparada en ese Producto de Usuario a partir de una versi\'{o}n 
modificada de su Fuente Correspondiente.  La informaci\'{o}n debe ser 
suficiente para garantizar que el funcionamiento continuado del c\'{o}digo 
fuente modificado no es prevenido o interferido por el simple hecho de 
que ha sido modificado.

Si usted transmite una obra en c\'{o}digo objeto bajo esta secci\'{o}n en, o 
con, o espec\'{i}ficamete para usar en, un Producto de Usuario, y la 
transmisi\'{o}n tiene lugar como parte de una transacci\'{o}n en la cual el 
derecho de posesi\'{o}n y uso de un Producto de Usuario es transferido a un 
destinatario en perpetuidad o por un periodo establecido 
(independientemente de c\'{o}mo se caracterice la operaci\'{o}n), la Fuente 
Correspondiente tranmitida bajo esta secci\'{o}n debe estar acompa\~nada de la 
Informaci\'{o}n de Instalaci\'{o}n.  Pero este requisito no se aplica si ni 
usted ni ning\'{u}n tercero tiene la capacidad de instalar c\'{o}digo objeto
modificado en el Producto de Usuario (por ejemplo, la obra ha sido 
instalada en la ROM).

El requisito de proveer de la Informaci\'{o}n de Instalaci\'{o}n no incluye 
el requisito de continuar proporcionando asistencia, garant\'{i}a, o 
actualizaciones para una obra que ha sido modificada o instalada por el 
destinatario, o para un Producto de Usuario en el cual ha sido 
modificada o instalada.  El acceso a una red puede ser denegado cuando 
la modificaci\'{o}n en s\'{i} afecta materialmente y adversamente el 
funcionamiento de la red o viola las reglas y protocolos de comunicaci\'{o}n 
de la red.

La Fuente Correspondiente transmitida, y la Informaci\'{o}n de Instalaci\'{o}n 
proporcionada, de acuerdo con esta secci\'{o}n debe estar en un formato que 
sea documentado p\'{u}blicamente (y con una implementaci\'{o}n disponible para 
el p\'{u}blico en formato de c\'{o}digo fuente), y no deben necesitar 
contrase\~nas o claves particulares para la extracci\'{o}n, lectura o copia.

\item T\'{e}rminos adicionales.

Los ``Permisos adicionales'' son t\'{e}rminos que se a\~naden a los t\'{e}rminos de 
esta Licencia haciendo excepciones de una o m\'{a}s de una de sus condiciones.
Los permisos adicionales que son aplicables al Programa entero deber\'{a}n ser tratados
como si estuvieran incluidos en esta Licencia, en la medida bajo la ley
aplicable. Si los permisos adicionales solo son aplicables a parte del
Programa, esa parte debe ser usada separadamente bajo esos permisos,
pero el Programa completo queda bajo la autoridad de esta Licencia sin
considerar los permisos adicionales.

Cuando se transmite una copia de una obra derivada, se puede
opcionalmente quitar cualesquiera permisos adicionales de esa copia,
o de cualquier parte de ella. Los permisos adicionales pueden ser
escritos para requerir su propia eliminaci\'{o}n bajo ciertas casos cuando
se modifica la obra. Se pueden colocar permisos adicionales en material,
a\~nadidos a una obra derivada, para los cuales se establecen o se 
pueden establecer los permisos de derechos de autor (``copyright'') apropiados.

No obstante cualquier otra disposici\'{o}n de esta Licencia, para el
material que se a\~nada a una obra derivada, se puede (si est\'{a} autorizado
por los titulares de los derechos de autor (``copyright'') del material) a\~nadir los t\'{e}rminos de esta 
Licencia con los siguientes t\'{e}rminos:
  \begin{enumerate}
  \item Ausencia de garant\'{i}a o limitaci\'{o}n de responsabilidad diferente de 
    los t\'{e}rminos de las secciones 15 y 16 de esta Licencia; o

  \item Exigir la preservaci\'{o}n de determinados avisos legales razonables o
     atribuciones de autor en ese material o en los Avisos Legales 
     Apropiados mostrados por los obras que lo contengan; o

  \item Prohibir la tergiversaci\'{o}n del origen de ese material, o requerir 
    que las versiones modificadas del material se marquen de maneras 
    razonables como diferentes de la versi\'{o}n original; o

  \item Limitar el uso con fines publicitarios de los nombres de los 
    licenciantes o autores del material; o

  \item Negarse a ofrecer derechos concedidos por leyes de registro para el
    uso de alguno nombres comerciales, marcas registradas o marcas de servicio; o

  \item Exigir la compensaci\'{o}n de los licenciantes y autores de ese material por 
    cualquiera que distribuya el material (o versiones modificadas del mismo) 
    estableciendo obligaciones contractuales de responsabilidad sobre el 
    destinatario, por cualquier responsabilidad que estas obligaciones 
    contractuales impongan directamente sobre los licenciantes y autores.
  \end{enumerate}

Todos los dem\'{a}s t\'{e}rminos adicionales no permisivos son consideradas 
``restricciones extra'' en el sentido de la secci\'{o}n 10.  Si el Programa, tal cual se 
recibi\'{o}, o cualquier parte del mismo, contiene un aviso indicando que se 
encuentra cubierto por esta Licencia junto con un t\'{e}rmino que es otra restricci\'{o}n,
se puede quitar ese t\'{e}rmino.  Si un documento de licencia contiene una restricci\'{o}n
adicional, pero permite relicenciar o redistribuir bajo esta Licencia, se puede 
a\~nadir a un material de la obra derivada bajo los t\'{e}rminos de ese documento de 
licencia, a condici\'{o}n de que dicha restricci\'{o}n no sobreviva el relicenciamiento 
o redistribuci\'{o}n.

Si se a\~naden t\'{e}rminos a una obra derivada de acuerdo con esta secci\'{o}n, se debe 
colocar, en los archivos fuente involucrados, una declaraci\'{o}n de los t\'{e}rminos adicionales aplicables a esos archivos, o un aviso indicando donde encontrar los t\'{e}rminos aplicables.

Las t\'{e}rminos adicionales, permisivos o no permisivos, pueden aparecer en forma de 
una licencia escrita por separado, o figurar como excepciones; los requisitos anteriores son aplicables en cualquier forma.

\item Conclusiones.

Usted no podr\'{a} propagar o modificar una obra amparada salvo lo
expresamente permitido por esta Licencia.  Cualquier intento
diferente de propagaci\'{o}n o modificaci\'{o}n ser\'{a} considerado nulo y
autom\'{a}ticamente se anular\'{a}n sus derechos bajo esta Licencia
(incluyendo las licencias de patentes concedidas bajo el tercer
p\'{a}rrafo de la secci\'{o}n 11).

Sin embargo, si usted deja de violar esta Licencia, entonces su
licencia de un titular de los derechos de autor (``copyright'') correspondiente ser\'{a}
restituida (a) provisionalmente, a menos que y hasta que el titular de los derechos de autor (``copyright'') expl\'{i}cita y finalmente termine tu licencia, y (b) permanentemente,
si el titular del copyright no le ha notificado su violaci\'{o}n por
alg\'{u}n medio razonable antes de los 60 d\'{i}as siguientes a la cesaci\'{o}n.

Adem\'{a}s, su licencia de un titular de los derechos de autor (``copyright'')
correspondiente ser\'{a} restituida permanentemente si el titular de los derechos de autor (``copyright'') le notifica la violaci\'{o}n por alg\'{u}n medio razonable,
siendo \'{e}sta la primera vez que recibe la notificaci\'{o}n de violaci\'{o}n de
esta Licencia (para cualquier obra) de ese titular de los derechos de autor (``copyright''), y
usted subsana la violaci\'{o}n antes de 30 d\'{i}as despu\'{e}s de la recepci\'{o}n
de la notificaci\'{o}n.

La cancelaci\'{o}n de sus derechos bajo esta secci\'{o}n no da por
canceladas las licencias de terceros que hayan recibido copias o
derechos de usted bajo esta Licencia.  Si sus derechos han sido
cancelados y no fueran renovados de manera permanente, usted no
cumple los requisitos para recibir nuevas licencias para el mismo
material bajo la secci\'{o}n 10.

\item Aceptaci\'{o}n No Obligatoria por Tenencia de Copias.

Usted no est\'{a} obligado a aceptar esta Licencia por recibir o
ejecutar una copia del Programa.  La propagaci\'{o}n adicional de una obra amparada surgida \'{u}nicamente como consecuencia de usar una transmisi\'{o}n peer-to-peer para recibir una copia tampoco requiere aceptaci\'{o}n.  Sin embargo, esta Licencia s\'{o}lo le otorga
permiso para propagar o modificar cualquier obra amparada.  Estas
acciones infringen los derechos de autor (``copyright'') si usted no acepta esta Licencia.  Por lo tanto, al modificar o distribuir una obra amparada, usted
indica que acepta esta Licencia para poder hacerlo.

\item Herencia Autom\'{a}tica de Licencia para Destinatarios.

Cada vez que transmita una obra amparada, el destinatario recibir\'{a} 
autom\'{a}ticamente una licencia de los licenciadores originales, para 
ejecutar, modificar y distribuir esa obra, sujeto a esa Licencia. 
Usted no ser\'{a} responsable de asegurar el cumplimiento de esta Licencia 
por terceros.

Una ``transacci\'{o}n de entidad'' es una transacci\'{o}n que transfiere el 
control de una organizaci\'{o}n, o sustancialmente todos los bienes de una, 
o subdivide una organizaci\'{o}n, o fusiona organizaciones. Si la 
propagaci\'{o}n de una obra amparado surge de una transacci\'{o}n de entidad, 
cada parte en esa transacci\'{o}n que reciba una copia de la obra tambi\'{e}n 
recibe todas las licencias de la obra que la parte interesada tuviese 
o pudiese ofrecer seg\'{u}n el p\'{a}rrafo anterior, adem\'{a}s del derecho a tomar 
posesi\'{o}n de las Fuentes Correspondientes de la obra a trav\'{e}s del 
predecesor interesado, si el predecesor tiene o puede conseguirla 
con un esfuerzo razonable.

Usted no podr\'{a} imponer ninguna restricci\'{o}n posterior en el ejercicio 
de los derechos otorgados o concedidos bajo esta Licencia. Por ejemplo, 
usted no puede imponer un pago por licencia, derechos u otros cargos 
por el ejercicio de los derechos otorgados bajo esta Licencia, y no 
puede iniciar litigios (incluyendo demandas o contrademandas en 
pleitos) alegando cualquier reclamaci\'{o}n de violaci\'{o}n de patentes por 
cambiar, usar, vender, ofrecer en venta o importar el Programa o 
alguna parte del mismo.

\end{enumerate}

\end{document}
